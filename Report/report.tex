\documentclass{sigchi}

% Use this section to set the ACM copyright statement (e.g. for
% preprints).  Consult the conference website for the camera-ready
% copyright statement.

% Copyright
%\CopyrightYear{2020}
%\setcopyright{acmcopyright}
%\setcopyright{acmlicensed}
%\setcopyright{rightsretained}
%\setcopyright{usgov}
%\setcopyright{usgovmixed}
%\setcopyright{cagov}
%\setcopyright{cagovmixed}
% DOI
%\doi{https://doi.org/10.1145/3313831.XXXXXXX}
% ISBN
% \isbn{978-1-4503-6708-0/20/04}
%Conference
\conferenceinfo{ACM'23,}{November  16--17, 2023, Gatlinburg, TN, USA}
%Price

% Use this command to override the default ACM copyright statement
% (e.g. for preprints).  Consult the conference website for the
% camera-ready copyright statement.

%% HOW TO OVERRIDE THE DEFAULT COPYRIGHT STRIP --
%% Please note you need to make sure the copy for your specific
%% license is used here!
% \toappear{
% Permission to make digital or hard copies of all or part of this work
% for personal or classroom use is granted without fee provided that
% copies are not made or distributed for profit or commercial advantage
% and that copies bear this notice and the full citation on the first
% page. Copyrights for components of this work owned by others than ACM
% must be honored. Abstracting with credit is permitted. To copy
% otherwise, or republish, to post on servers or to redistribute to
% lists, requires prior specific permission and/or a fee. Request
% permissions from \href{mailto:Permissions@acm.org}{Permissions@acm.org}. \\
% \emph{CHI '16},  May 07--12, 2016, San Jose, CA, USA \\
% ACM xxx-x-xxxx-xxxx-x/xx/xx\ldots \$15.00 \\
% DOI: \url{http://dx.doi.org/xx.xxxx/xxxxxxx.xxxxxxx}
% }

% Arabic page numbers for submission.  Remove this line to eliminate
% page numbers for the camera ready copy
% \pagenumbering{arabic}

% Load basic packages
\usepackage{balance}       % to better equalize the last page
\usepackage{graphics}      % for EPS, load graphicx instead 
\usepackage[T1]{fontenc}   % for umlauts and other diaeresis
\usepackage{txfonts}
\usepackage{mathptmx}
\usepackage[pdflang={en-US},pdftex]{hyperref}
\usepackage{color}
\usepackage{booktabs}
\usepackage{textcomp}


% Some optional stuff you might like/need.
\usepackage{microtype}        % Improved Tracking and Kerning
% \usepackage[all]{hypcap}    % Fixes bug in hyperref caption linking
\usepackage{ccicons}          % Cite your images correctly!
% \usepackage[utf8]{inputenc} % for a UTF8 editor only

% If you want to use todo notes, marginpars etc. during creation of
% your draft document, you have to enable the "chi_draft" option for
% the document class. To do this, change the very first line to:
% "\documentclass[chi_draft]{sigchi}". You can then place todo notes
% by using the "\todo{...}"  command. Make sure to disable the draft
% option again before submitting your final document.
\usepackage{todonotes}

% Paper metadata (use plain text, for PDF inclusion and later
% re-using, if desired).  Use \emtpyauthor when submitting for review
% so you remain anonymous.
\def\plaintitle{Revitalizing Solar Insights: A Dashboard for West Tennessee Solar Farm}
\def\plainauthor{Joshua Chamberlain, Andy Lum}
\def\emptyauthor{}
\def\plainkeywords{Authors' choice; of terms; separated; by
  semicolons; include commas, within terms only; this section is required.}
\def\plaingeneralterms{Documentation, Standardization}

% llt: Define a global style for URLs, rather that the default one
\makeatletter
\def\url@leostyle{%
  \@ifundefined{selectfont}{
    \def\UrlFont{\sf}
  }{
    \def\UrlFont{\small\bf\ttfamily}
  }}
\makeatother
\urlstyle{leo}

% To make various LaTeX processors do the right thing with page size.
\def\pprw{8.5in}
\def\pprh{11in}
\special{papersize=\pprw,\pprh}
\setlength{\paperwidth}{\pprw}
\setlength{\paperheight}{\pprh}
\setlength{\pdfpagewidth}{\pprw}
\setlength{\pdfpageheight}{\pprh}

% Make sure hyperref comes last of your loaded packages, to give it a
% fighting chance of not being over-written, since its job is to
% redefine many LaTeX commands.
\definecolor{linkColor}{RGB}{6,125,233}
\hypersetup{%
  pdftitle={\plaintitle},
% Use \plainauthor for final version.
%  pdfauthor={\plainauthor},
  pdfauthor={\emptyauthor},
  pdfkeywords={\plainkeywords},
  pdfdisplaydoctitle=true, % For Accessibility
  bookmarksnumbered,
  pdfstartview={FitH},
  colorlinks,
  citecolor=black,
  filecolor=black,
  linkcolor=black,
  urlcolor=linkColor,
  breaklinks=true,
  hypertexnames=false
}

% create a shortcut to typeset table headings
% \newcommand\tabhead[1]{\small\textbf{#1}}

% End of preamble. Here it comes the document.
\begin{document}

\title{\plaintitle}

\numberofauthors{2}
\author{%
  \alignauthor{Joshua Chamberlain\\
    \affaddr{Martin, TN}\\
    \email{jospcham@ut.utm.edu}}\\
  \alignauthor{Andy Lum\\
    \affaddr{Lakeland, TN}\\
    \email{andlum@ut.utm.edu}}\\
}

\maketitle
\begin{abstract}
This project we constructed aims to be an interactive dashboard for displaying solar irradiance data collected at a photovoltaic power station. Given a recent push by the University of Tennessee Research Foundation toward revitalizing its use, the West Tennessee Solar Farm will serve as a template. This location is of particular interest due to its proximity to Blue Oval City (the site of the new Ford manufacturing plant, near Stanton, TN). With the farm’s existing dashboard being non-functional, there is a demand for a solution, which we will achieve through MySQL, Python, Google Drive API, R-Shiny, Shinyapps.io, and Google Cloud Console.
MySQL serves as our data hub, efficiently organizing solar energy data by sensor location. Python, coupled with the Google Drive API, simulates real-time data collection. 

The core of the project is an R-Shiny dashboard offering real-time data visualization, interactive maps, detailed sensor information, and access to historical data and analysis. Users can select their desired time frames. Shinyapps.io hosts the dashboard, ensuring accessibility across diverse platforms, such as web browsers and various operating systems. This approach allows users from all major operating systems to access the dashboard, promoting widespread accessibility. To further fortify data security and enhance user convenience, Google Cloud Console safeguards our API information.
Our dashboard incorporates an export function, enabling users to extract data. In addition, we constructed an easy-to-use webpage that is accessible across various major operating systems. This approach ensures that our project is widely available and caters to a diverse audience; thus, making valuable solar irradiance data easily accessible to all. This project aims to provide researchers, policymakers, and the public with real-time insights into solar irradiance data at the West Tennessee Solar Farm, supporting sustainable energy solutions.


\end{abstract}

\section{Introduction}

In the pursuit of sustainable energy solutions, our project ventures to unlock the vast potential of solar irradiance data. Our focus is on the West Tennessee Solar Farm, situated near Stanton, TN \cite{WTSF}. Its strategic proximity to Blue Oval City adds notable importance. The critical challenge we address is the current, non-functional existing dashboard for the solar farm, which has left vital solar irradiance data inaccessible for potential users. The West Tennessee Solar Farm offers the potential for abundant clean energy, but its potential remains unrealized in part due to the limitations of the existing dashboard. In this project, we aspire to bridge this data accessibility gap while opening up opportunities to inform and educate the general public about the farm's renewable energy potential.

In response to this need, our project emerges as a symbol of innovation and accessibility, seamlessly integrating an array of cutting-edge technologies. With MySQL as our data hub, we thoroughly organize solar energy data by sensor location. Python, in collaboration with the Google Drive API, takes on the role of capturing and decoding real-time data with unwavering precision. However, the importance of our project lies in R-Shiny dashboard, an interactive interface that goes beyond the typical data visualization. It not only brings solar irradiance data to life in real-time but also empowers users with interactive maps, intricate sensor insights, information that is relatable to the lay-person, and a comprehensive historical archive.

\section{Motivation}

The West Tennessee Solar Farm, located near Stanton, TN, represents a beacon of hope in the quest for clean and sustainable energy production. Nevertheless, despite its significant potential, the realization of this promise has remained elusive due to the operational constraints of the existing dashboard. Unfortunately, the dashboard has remained non-functional, thereby preventing users from accessing vital solar irradiance data. As a result, the solar farm's ability to significantly contribute to the renewable energy sector has been hampered.

This challenge serves as the stimulus for our proactive engagement with this project. Our project was further motivated by the valuable input we received from the CEO and President of the West Tennessee Solar Farm during a conference with Dr. Sims. When we presented our initial dashboard, they were highly impressed with the concept and functionality. Their constructive feedback propelled us to further refine and develop the final dashboard product that we proudly offer today.

Our interactive dashboard not only addresses the immediate needs of the solar farm but also extends its utility to education and research purposes. With this tool, we enable students, researchers, and the general public to gain insightful access to real-time solar irradiance data. This not only empowers them to better understand and appreciate the significance of renewable energy but also fosters a collaborative environment where valuable insights can be shared and applied. Whether for educational institutions or research facilities, our interactive dashboard serves as an invaluable resource for learning, innovation, and informed decision-making in the pursuit of sustainable energy solutions.


\section{Technologies Used}

\subsection{MySQL}
The MySQL database was used to store information about irradiance collected from data sensors. The database schema included a main table called Irradiance that contained the following information:
\begin{itemize}
\item[1.] The day of year the data was recorded in, ranging from 1 - 365 representing each day of the year.
\item[2.] The minute of the day that the data was recorded in, ranging from 1 - 1440 representing each minute of the day.
\item[3.] 10 separate columns containing the collected irradiance from each sensor in watts per meter squared ($\dfrac{W}{m^2}$). 
\item[4.] The total irradiance of all sensors.
\end{itemize}

\subsection{Python}
The python program uses the mysql library\cite{MySQL} to connect to the database and retrieve data from the database. It gets the information at a rate of 1 minute, providing real time updates to the R-Shiny dashboard. It does this by creating a query that is then executed on the database.

\subsection{R-Shiny}
Through multiple packages, R-Shiny is used to perform statistical analyses and to plot both power production and irradiance. It also holds the code for the user interface and performs the conversion between irradiance and power production. 

\subsection{ShinyApps.io} 
ShinyApps.io is used to host the dashboard as a website on the cloud. It gives access to the dashboard for the public. This is essential since one of the main goals of the project is to make it an educational tool as well as a research tool. It uses the Posit Cloud, which is the main cloud server for R-Shiny programs.

\subsection{Google Cloud Console}
Google Cloud Console is used to secure the data being transferred to the website. It protects it by masking passwords and API keys to the database and the APIs being used.

\section{R Packages}
R packages play a pivotal role in bringing the solar farm data dashboard to life. R packages are collections of functions, documentation, and sample data bundled together for specific tasks or purposes. In the project, several key R packages have been harnessed to create a robust and interactive dashboard, catering to the real-time needs of the solar farm data:

\subsection{Shiny}

Shiny is an R package used to bridge the gap between R and HTML. It was used to design the UI of the dashboard as well as connecting the data collected from the MySQL database to the R program. Finally, it allowed reactivity in the R program, making real time data visualization possible.\cite{R-Shiny}\cite{Wickham_2021}

\subsection{Leaflet}

Leaflet is used to provide the map and all of its functionalities. It uses the satellite photos from the USGS (United States Geological Survey) for the interactive picture of the solar farm. Leaflet also contributed to the popups and zoom features of the map, allowing the user to have easy access to information from each sensor.\cite{Leaflet}

\subsection{OpenMeteo}

OpenMeteo is an API used for weather forecasting and historical data analysis. It allowed for forecasting the next day on the dashboard. It provided the hourly cloud cover, hourly temperature, weather code, and the amount of shortwave UV radiation. The hourly variables were averaged for user convenience.\cite{OpenMeteo}

\subsection{Google Drive}
The google drive package allows the program to connect to google drive files. It is used to connect to the csv file that contains all the processed data for the dashboard plots and sensor information.\cite{GD}

\subsection{Rsconnect}
The rsconnect package connects the R-shiny application to shinyapps.io server. \cite{RSConnect}

\subsection{GGPlot2 and Plotly}
GGPlot2 and Plotly allows for the interactivity in the plots. It allows the user to hover over the plot and see the minute the data was gathered and the irradiance at that time. It also allows null data to be absent from the dataset. 

\section{Initial Mockup}
\includegraphics[scale = 0.3]{Initial Mockup.pdf}
\label{fig:initial-mockup}


\subsection{Initial Features}
The initial features of the first mockup includes:
\begin{itemize}
    \item A title for the webpage
    \item A map with sensor markers and Popups that display the current irradiance
    \item An irradiance plot that shows data in a selected time frame reactively
    \item A power production plot that takes the irradiance and converts it into power production reactively.
    \item A data export button that allows the user to download data for a specified time frame
    \item A date selector that allows the user to specify the time of the data shown
    \item A button that allows the user to switch between plotting daily, weekly, and monthly data plots.
\end{itemize}

\section{Finalized Dashboard Features}
Throughout the design process many features have either been added or removed.
\subsection{Finalized Dashboard}
\includegraphics[scale = 0.18]{Thumbnail.png}
\subsection{Reactive Plots}
The dashboard's main feature is the reactive plots that show the current irradiance level or power production. The plots are able to change from live plot to daily plots. Also within the daily plots, the plots can switch between irradiance and power production. It does this using a formula gathered through scientific study.\cite{PowerConversion}

\subsection{Interactive Map}
The map that shows the West Tennessee Solar Farm\cite{WTSF} from an ariel view. It has clickable sensors that pop-up with the current irradiance value at that time. The sensors change colors when one is abnormally low or abnormally high. It is colored based on a red to green gradient.\ref{fig:initial-mockup} The map also has the capability to zoom in and out. All of the interactivity and reactivity within the map was made using the Leaflet package.\cite{Leaflet}

\subsection{Weather Forecast Table}
The dashboard has a weather forecast for the next day. It gives the cloud cover, temperature, weather code, and total predicted shortwave ultraviolet radiation. This allows not only users to know the predicted forecast, it also allows the West Tennessee Solar Farm to plan storage based on how much power the farm will have to supplement the main power supply.

\subsection{Exporting Daily Data}
The dashboard has a button to download the specific days data the user is looking at as a csv. This is achieved using the downloadHandler in Shiny.\cite{R-Shiny} Pressing the button will pop up with a "Save As" screen on the users local computer. By default it will name the file in the format, mm-dd-yyyy.csv, where the date is the current days plot the user is looking at.

\subsection{Educational Features}
The dashboard has two educational sections. The first section is a description of what irradiance is. The second section contains reactive text that converts the total power production to more relatable units: amount of houses powered and how many electric cars are being powered. This calculated for the day, week, and month.

\section{Other Features}
Along with the dashboard there are two other tabs.
\subsection{Solar Power Process}
The tab explains in a simple format how the solar panels collect sunlight and convert it to power. It uses the fade in CSS animation to keep users engaged. It is meant to serve as an education and research tool for the public outreach and awareness.
\subsection{Frequently Asked Questions}
The tab answers Frequently Asked Questions about solar farms and the solar power process. 

\section{Power Conversion Formula}
The dashboard has the functionality for the data to be converted from irradiance in $\dfrac{W}{m^2}$ to total power production in $kWh$. Due to certain factors, such as the angle of the rays, the angle of the panel, the amount of interference and a particular solar's farm efficiency factor, the irradiance to power production conversion is fascinating formula. Through a study in the \emph{Renewable and Sustainable Energy Reviews} in 2022\cite{PowerConversion}, found that taking into all of these factors, and simplifying it down to a reasonable level the formula was as follows: Power Production $\approx -0.038 + 0.0058 * Current Irradiance$. For lower values this would result in a measure less than 0. We com batted this by using the following R code: max(0, -0.038 + 0.0058 * Current Irradiance). This data assumes that the angle of the solar panel, the angle of the UV rays, and the solar farm efficiency factor has been taken into account. 

% ``Quotation Example''

%\section{Accessibility}
%The Executive Council of SIGCHI has committed to making SIGCHI
%conferences more inclusive for researchers, practitioners, and
%educators with disabilities. As a part of this goal, the all authors
%are asked to work on improving the accessibility of their
%submissions. Specifically, we encourage authors to carry out the
%following five steps:
%\begin{enumerate}
%\item Add alternative text to all figures
%\item Mark table headings
%\item Add tags to the PDF
%\item Verify the default language
%\item Set the tab order to ``Use Document Structure''
%\end{enumerate}
%For more information and links to instructions and resources, please
%see: \url{http://chi2016.acm.org/accessibility}.  The
%\texttt{{\textbackslash}hyperref} package allows you to create well tagged PDF files,
%please see the preamble of this template for an example.


\section{Conclusion}

In conclusion, our project represents more than just a solution to an immediate challenge. It embodies a valuable educational resource, catering to individuals keen on delving into the intricacies of solar farm operations and the utilization of renewable energy sources. With our interactive dashboard, learners can gain a comprehensive understanding of the solar farm process, from real-time data access to interactive map exploration, and even insights tailored to the layperson. 

Additionally, our work extends its reach to benefit researchers by providing them with unfettered access to public data essential for their research. This research-friendly feature allows academics and scientists to harness the potential of solar irradiance data, thereby contributing to the field's growth and development.

By presenting our innovative and accessible solution, we've not just stated our intentions but have demonstrated a tangible contribution to the realm of renewable energy. Readers will discover a tool that bridges the gap between renewable energy potential and reality, making clean energy more accessible, understandable, and usable. Our project aims to be a valuable resource for those looking to learn and research in this area, offering a unique blend of functionality that brings the West Tennessee Solar Farm and its data to life for the benefit of education and research alike.

\section{Acknowledgments}

We would like to thank all of our partners for helping us make this project possible. Thank you to:
\begin{itemize}
    \item Dr. Sims
    \item West Tennessee Solar Farm
    \item University of Tennessee Research Foundation
    \item University of Tennessee at Martin Department of Computer Science and
    \item University of Tennessee at Martin Department of Mathematics and Statistics
\end{itemize}
\balance{}

% BALANCE COLUMNS
\balance{}

% REFERENCES FORMAT
% References must be the same font size as other body text.
%%\bibliographystyle{SIGCHI-Reference-Format}
%% \bibliographystyle{acm-sigchi-proceedings}

\bibliographystyle{plain}
\bibliography{bibliography}

\end{document}
